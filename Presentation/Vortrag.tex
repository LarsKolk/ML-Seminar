\PassOptionsToPackage{unicode}{hyperref}
\documentclass[aspectratio=1610, 9pt]{beamer}

% Load packages you need here
%\usepackage{polyglossia}
%\setmainlanguage{german}

\usepackage{csquotes}

\usepackage{amsmath}
\usepackage{amssymb}
\usepackage{mathtools}

\usepackage{braket}
\usepackage{graphicx}

\usepackage{hyperref}
\hypersetup{
  linkcolor= {tudark}, % internal links
  citecolor={tugreen}, % citations
  urlcolor={tudark} % external links/urls
  }
\usepackage{bookmark}

\usepackage[english]{babel}
\usepackage[
backend=biber,
style=authoryear-comp
]{biblatex}

\bibliography{lit.bib}

\usepackage{siunitx}
\usepackage{multicol}

\usepackage{booktabs}

\definecolor{light-gray}{HTML}{b0b5b0}

% load the theme after all packages

\usetheme[
  showtotalframes, % show total number of frames in the footline
]{tudo}

% Put settings here, like
\unimathsetup{
  math-style=ISO,
  bold-style=ISO,
  nabla=upright,
  partial=upright,
  mathrm=sym,
}

\title{Vorhersage des wahrscheinlichsten Gewinner in dem Spiel Blackjack}
\author[L.~Kolk,~J.~Blank]{Lars Kolk\\ Jonah Blank}
\institute[ML-Seminar]{\\[0.3cm]TU Dortmund \\ \Large ML-Seminar}

\begin{document}



\maketitle


\begin{frame}{Fragestellung}
\begin{itemize}
  \item Grundlegende Fragestellung:\\
  \rightarrow{} ``Wie hoch ist die Wahrscheinlichkeit dass Spieler $x$ das Spiel nach $y$ ausgeteilten Karten gewinnt?''
  \vspace{0.5cm}
  \item Weitere möglichen Untersuchungsschwerpunkte:
  \begin{itemize}
    \item -
  \end{itemize}
\end{itemize}



\end{frame}


\begin{frame}{Datensatz}
    \begin{itemize}
    \item Quelle: \href{https://www.kaggle.com/mojocolors/900000-hands-of-blackjack-results}{Kaggle}
    \item Lizenz: Public Domain (CC0)
    \item $150000$ Spiele
    %Variablen - dein Part Jonah 
    \end{itemize}
\end{frame}

\begin{frame}{Methode zur Klassifikation und Vergleich mit Alternativen}
  \begin{itemize}
    \item Lerne Blackjack-Strategie \\
    \rightarrow{} Nutze reinforced learning\\
    \item Mögliche einfache Vergleichsmethode:
    \begin{enumerate}
      \item Kartenzählen
      \item Wahrscheinlichkeitsrechnungen
    \end{enumerate}
    \item Kann ein Neuronales Netz Blackjack lernen?
  \end{itemize}

\end{frame}



\end{document}

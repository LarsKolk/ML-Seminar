\PassOptionsToPackage{unicode}{hyperref}
\documentclass[aspectratio=1610, 9pt]{beamer}

% Load packages you need here
%\usepackage{polyglossia}
%\setmainlanguage{german}

\usepackage{csquotes}

\usepackage{amsmath}
\usepackage{amssymb}
\usepackage{mathtools}

\usepackage{braket}
\usepackage{graphicx}

\usepackage{hyperref}
\hypersetup{
  linkcolor= {tudark}, % internal links
  citecolor={tugreen}, % citations
  urlcolor={tudark} % external links/urls
  }
\usepackage{bookmark}

\usepackage[english]{babel}
\usepackage[
backend=biber,
style=authoryear-comp
]{biblatex}

\bibliography{lit.bib}

\usepackage{siunitx}
\usepackage{multicol}

\usepackage{booktabs}

\definecolor{light-gray}{HTML}{b0b5b0}

% load the theme after all packages

\usetheme[
  showtotalframes, % show total number of frames in the footline
]{tudo}

% Put settings here, like
\unimathsetup{
  math-style=ISO,
  bold-style=ISO,
  nabla=upright,
  partial=upright,
  mathrm=sym,
}

\title{Vorhersage des wahrscheinlichsten Gewinner in dem Spiel Blackjack}
\author[L.~Kolk,~J.~Blank]{Lars Kolk\\ Jonah Blank}
\institute[ML-Seminar]{\\[0.3cm]TU Dortmund \\ \Large ML-Seminar}

\begin{document}



\maketitle


\begin{frame}{Fragestellung}
\begin{itemize}
  \item Grundlegende Fragestellung:\\
  \rightarrow{} ``Wie hoch ist die Wahrscheinlichkeit dass ein bestimmter Spieler das Spiel nach $y$ Zügen gewinnt?''
  \vspace{0.5cm}
  \item Motivation
  \begin{itemize}
    \item Blackjack ist ein Glücksspiel
    \begin{itemize}
      \item Wie aus Filmen bekannt kann das "Zählen von Karten" einen großen Vorteil bringen
      \item Kann ein neuronales Netzwerk auch Vorhersagen, wie wahrscheinlich man gewinnt?
      \item Ist es schneller oder zuverlässiger?
    \end{itemize}
  \end{itemize}
\end{itemize}
\end{frame}


\begin{frame}{Datensatz}
    \begin{itemize}
    \item Quelle: \href{https://www.kaggle.com/mojocolors/900000-hands-of-blackjack-results}{Kaggle}
    \item Lizenz: Public Domain (CC0)
    \item $900000$ Hands
    \begin{itemize}
    \item $6$ Spieler(\texttt{PlayerNo}) + $1$ Kroupier
	\item Jeder Spieler bekommt am Anfang zwei Karten mit Wert \texttt{ply2cardsum} und setzt Geld 
    \item Es werden maximal 5 Karten an jeden ausgeteilt mit Wert \texttt{card[i]}/\texttt{dealcard[i]}
    \item Exakt 21 Punkte? \texttt{blkjck}, gewonnen? \texttt{winloss}
    \item Geschlagen oder \glqq busted \grqq\, \texttt{plybustbeat}/\texttt{dlbustbeat}
    \item Geldgewinn \texttt{plwinamt}/\texttt{plwinamt}\\
    \rightarrow() mögliche Zusatzfrage: Ist die Methode mit der größten Siegeswahrscheinlichkeit auch die mit dem größten Gewinn?
    \end{itemize}
    \end{itemize}
\end{frame}

\begin{frame}{Methode zur Klassifikation und Vergleich mit Alternativen}
  \begin{itemize}
    \item Lerne Blackjack-Strategie \\
    \rightarrow{} Nutze reinforced learning\\
    \item Mögliche einfache Vergleichsmethode:
    \begin{enumerate}
      \item Kartenzählen
      \item Wahrscheinlichkeitsrechnungen
    \end{enumerate}
    \item Kann ein Neuronales Netz Blackjack lernen?
  \end{itemize}
\end{frame}



\end{document}
